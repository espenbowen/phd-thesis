\section{Introduction}
\label{sec:intro}

Particle physics is the study of the basic constituents of matter and their interactions, and aims to describe the fundamental laws governing the nature of the physical universe. It is a vast, diverse field encompassing both theoretical and experimental communities.

This thesis will focus on the \lhcb experiment, situated at the Large Hadron Collider (\lhc) at \cern. The \lhc is the world's largest and most powerful particle accelerator and represents the forefront of experimental particle physics research. \lhcb is one of several detector experiments designed to study the debris produced by the colliding beams of particles.

Chapter~\ref{sec:theory} provides a theoretical motivation for the experimental studies presented in later chapters. A brief introduction to the framework of modern particle physics is given, followed by a description of the properties of the decays of rare \B mesons.

Chapter~\ref{sec:lhcb} describes the \lhcb experiment, both current and future, with emphasis given to the tracking sub-detectors and trigger system.

Chapter~\ref{sec:track} introduces track reconstruction in \lhcb. A description of each of the tracking algorithms employed during \lhcb Run I is provided, along with an explanation of how tracking performance is characterised.

Chapter~\ref{sec:up-track-upgrade} details the development of an improved upstream tracking algorithm for use in the reconstruction sequence of the \lhcb Upgrade trigger. The motivation and initial performance are given, followed by a comprehensive description of the improvements made to the algorithm and the subsequent gains in performance. 

Chapter~\ref{sec:up-track-run2} describes how the upstream tracking algorithm was subsequently adapted for use in the reconstruction sequence of the \lhcb Run II trigger and the improvements in performance achieved.

Chapter~\ref{sec:kpimm} describes the measurement of the differential branching and angular moments analysis of \BdToKpimm decays in the $K^{*}_{0,2}(1430)^{0}$ region. The dataset used has been collected by the \lhcb experiment in $pp$ collisions at centre of mass energies of 7 and 8${\mathrm{\,Te\kern -0.1em V}}$, corresponding to an integrated luminosity of 3$\mbox{\,fb}^{-1}$.