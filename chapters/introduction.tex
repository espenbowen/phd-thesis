\section{Introduction}
\label{sec:intro}


Chapter~\ref{sec:theory} provides a theoretical motivation for the experimental studies presented in later chapters. A brief  introduction to the framework of modern particle physics is given, followed by a description of the properties of the decays of rare \B mesons.

Chapter~\ref{sec:lhcb} describes the \lhcb experiment, both current and future, with emphasis given to the tracking sub-detectors and trigger system.

Chapter~\ref{sec:track} introduces track reconstruction in \lhcb. A description of each of the tracking algorithms employed during \lhcb Run I is provided, along with an explanation of how tracking performance is characterised.

Chapter~\ref{sec:up-track-upgrade} details the development of an improved upstream tracking algorithm for use in the reconstruction sequence of the \lhcb Upgrade trigger. The motivation and initial performance are given, followed by a comprehensive description of the improvements made to the algorithm and the subsequent gains in performance. 

Chapter~\ref{sec:up-track-run2} describes how the upstream algorithm was subsequently adapted for use in the reconstruction sequence of the \lhcb Run II trigger and the improvements in performance achieved.

Chapter~\ref{sec:kpimm} describes the measurement of the differential branching and angular moments analysis of \BdToKpimm decays in the in the $K^{*}_{0,2}(1430)^{0}$ region. The dataset used has been collected by the LHCb experiment in $pp$ collisions at centre of mass energies of 7 and 8${\mathrm{\,Te\kern -0.1em V}}$, corresponding to an integrated luminosity of 3$\mbox{\,fb}^{-1}$.