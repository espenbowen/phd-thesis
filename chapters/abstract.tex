\section*{Abstract}

The LHCb detector is a single-arm forward spectrometer covering the pseudorapidity range $2 < \eta < 5$, designed to search for indirect evidence of New Physics in $C\!P$ violation and rare decays of beauty and charm hadrons. The unique geometry takes advantage of the large $b$- and $c$-quark production in the forward region at the LHC. The detector includes a high granularity silicon-strip vertex detector, a silicon-strip detector upstream of the magnet and three stations of silicon-strip detectors and straw drift tubes downstream of the magnet.

This thesis is divided into two main parts. The first part details the development of improved algorithms to perform track reconstruction using the sub-detectors upstream of the LHCb magnet. A novel idea to perform upstream tracking as an intermediate step of the track reconstruction sequence was investigated. The vast gains in tracking performance obtained when using upstream tracks led to the algorithm being adopted into the default reconstruction sequence for the LHCb Upgrade. It will play a large role in allowing LHCb to become the first hadron collider experiment to operate a software-only trigger at the full event rate. Following the success of upstream tracking for the Upgrade scenario, a similar strategy was developed for LHCb Run II. The resulting algorithm was included in the tracking sequence in the first stage of the software trigger, greatly improving the signal efficiency for charm physics and allowing lifetime unbiased triggers for hadronic final states for the first time.

The second part of this thesis describes the measurements of the differential branching fraction and angular moments of the decay $\decay{\B^0}{K^{+}\pi^{-}\mu^{+}\mu^{-}}$ in the $K^{*}_{0,2}(1430)^{0}$ region. Proton-proton collision data, corresponding to an integrated luminosity of 3$\mbox{\,fb}^{-1}$ collected by the LHCb experiment, are used. Differential branching fraction measurements are reported in five bins of the invariant mass squared of the dimuon system, \qsq, between $0.1$ and $8.0{\mathrm{\,Ge\kern -0.1em V^2\!/}c^4}$. For the first time, an angular analysis of the S-, P- and D-wave contributions of this rare decay is performed. The set of 40 normalised angular moments that describe the decay is presented for the \qsq range $1.1$--$6.0{\mathrm{\,Ge\kern -0.1em V^2\!/}c^4}$. 