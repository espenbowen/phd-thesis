\section*{Abstract}

The LHCb detector is a single-arm forward spectrometer covering the pseudorapidity range $2 < \eta < 5$, designed to search for indirect evidence of New Physics in $C\!P$ violation and rare decays of beauty and charm hadrons. The unique geometry takes advantage of the large $b$ and $c$ quark production in the forward region at the LHC. The detector includes a high granularity silicon-strip vertex detector, a silicon-strip detector upstream of the magnet and three stations of silicon-strip detectors and straw drift tubes downstream of the magnet.

This thesis is divided into two main parts. The first part details the development of improved algorithms to perform track reconstruction using the sub-detectors upstream of the LHCb magnet. A novel idea to perform upstream tracking as an intermediate step of the track reconstruction sequence was investigated. The vast gains in tracking performance obtained when using upstream tracks led to the algorithm being adopted into the default reconstruction sequence for the LHCb Upgrade. It will play a large role in allowing LHCb to become the first hadron collider experiment to operate a software-only trigger at the full event rate. Following the success of upstream tracking for the Upgrade scenario, a similar strategy was developed for LHCb Run 2. The resulting algorithm was included in the tracking sequence in the first stage of the software trigger, greatly improving the signal efficiency for charm physics and allowing lifetime unbiased triggers for hadronic final states for the first time.

The second part of this thesis describes the measurements of the differential branching fraction and angular moments of the decay $B^{0} \to K^{+}\pi^{-}\mu^{+}\mu^{-}$ in the $K^{*}_{0,2}(1430)^{0}$ region. Proton-proton collision data are used, corresponding to an integrated luminosity of 3$\mbox{\,fb}^{-1}$ collected by the LHCb experiment. Differential branching fraction measurements are reported in five bins of the invariant mass squared of the dimuon system, $q^{2}$, between $0.1$ and $8.0{\mathrm{\,Ge\kern -0.1em V^2\!/}c^4}$. For the first time, an angular analysis that is sensitive to the S-, P- and D-wave contributions of this rare decay is performed. The set of 40 normalised angular moments describing the decay is presented for the $q^{2}$ range $1.1$--$6.0{\mathrm{\,Ge\kern -0.1em V^2\!/}c^4}$. 

\clearpage

\section*{Zusammenfassung}

Der LHCb Detektor ist ein einarmiges Magnetspektrometer, welches den
Pseudorapidit{\"a}tsbereich $2 < \eta < 5$ abdeckt und f{\"u}r die Suche nach
indirekten Anzeichen neuer Physik in $C\!P$~verletzenden und seltenen
Zerf{\"a}llen optimiert ist. Die besondere Geometrie des Detektors nutzt den
gro{\ss}en Wirkungsquerschnitt f{\"u}r die Produktion von $b$~und $c$~Quarks
in der Vorw{\"a}rtsrichtung am LHC besonders gut aus. Ein wichtiger Teil des
Detektors dient der Rekonstruktion der Spuren und Impulse geladener Teilchen
und besteht aus einem hochaufl{\"o}senden Siliziumstreifen-Vertexdetektor,
einem Siliziumstreifen-Spurdetektor vor dem Spektrometermagneten und drei
Spurdetektoren aus Siliziumstreifen und Driftr{\"o}hrchen hinter dem Magneten.

Diese Arbeit besteht aus zwei Hauptteilen. Der erste Teil der Arbeit behandelt
die Entwicklung verbesserter Algorithmen zur Rekonstruktion der Trajektorien
geladener Teilchen in den Spurdetektoren vor dem Magneten. Ein neuartiger
Ansatz wurde entwickelt, der die Rekonstruktion von Spursegmenten vor dem
Magneten als Zwischenschritt in der Spurrekonstruktion benutzt. Es wurde
demonstriert, dass dieser Ansatz eine deutliche Verbesserung in der
Leistungsst{\"a}rke der Spurrekonstruktion erm{\"o}glicht, und dieser Ansatz
wurde daraufhin als fester Bestandteil in die Rekonstruktionssequenz f{\"ur}
den LHCb Upgrade integriert. Er wird einen wichtigen Beitrag dazu leisten,
dass LHCb als erstes Experiment an einem Hadronen-Beschleunigerring in der
Lage sein wird, die Datenselektion ausschliesslich auf Softwarealgorithmen
beruhen zu lassen, welche die Daten mit der vollen Ereignisrate des
Beschleunigers verarbeiten werden k{\"o}nnen.  Aufbauend auf der erfolgreichen
Entwicklung dieses Algorithmus f{\"u}r den LHCb Upgrade wurde eine analoge
Strategie f{\"u}r die momentan laufende zweite Datennahmeperiode des LHCb
Experiments entwickelt. Der resultierende Algorithmus wurde erfolreich in die
Rekonstruktionssequenz im ersten Niveau der Software-Triggerselektion
integriert, was zu einer signifikanten Verbesserung der Signaleffizienz
f{\"ur} Charmphysik gef{\"u}hrt hat und es zum ersten Mal erm{\"o}glicht hat,
Selektionen f{\"u}r Zerf{\"a}lle in rein hadronische Endzust{\"a}nde zu
entwickeln, welche keine Verzerrung der Verteilung der Zerfallszeiten
verursachen.

Der zweite Teil dieser Arbeit beschreibt die Messung des differenziellen
Verzweigungsverh{\"a}ltnisses und der Momente der Winkelverteilungen f{\"u}r
den Zerfall $B^{0} \to K^{+}\pi^{-}\mu^{+}\mu^{-}$ in der Region der
$K^{*}_{0,2}(1430)^{0}$-Resonanzen. Die Analyse basiert auf den Daten von
Proton-Proton Kollisionen im Umfang einer integrierten Luminosit{\"a}t von
3$\mbox{\,fb}^{-1}$, welche vom LHCb Experiment gesammelt wurden.
Differenzielle Verzweigungsverh{\"a}ltnisse wurden in f{\"u}nf Intervallen der
quadrierten invarianten Masse $q^{2}$\ des Zwei-M{\"u}onen-Systems zwischen $0.1$
and $8.0{\mathrm{\,Ge\kern -0.1em V^2\!/}c^4}$ bestimmt. Zum ersten Mal wurde
eine Analyse der Winkelverteilungen in diesem selten Zerfall durchgef{\"u}hrt,
welche auf Beitr{\"a}ge von S-, P- and D-Wellen sensitiv ist. Der komplette
Satz der 40~normalisierten Momente, welche die Winkelverteilung in diesem
Zerfall beschreiben, wurde f{\"u}r den $q^{2}$-Bereich
$1.1$--$6.0{\mathrm{\,Ge\kern -0.1em V^2\!/}c^4}$ bestimmt.