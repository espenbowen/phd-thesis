\subsection{Differential branching fraction}
\label{sec:kpimm:bf}
 
The differential branching fraction $\deriv\BF/\deriv\qsq$ of the decay \BdToKpimm in an interval ($q^{2}_{\text{min}}$, $q^{2}_{\text{max}}$) is given by

\begin{equation}
\begin{split}
\frac{\deriv\BF}{\deriv\qsq} = \frac{1}{(q^{2}_{\text{max}} - q^{2}_{\text{min}}) }&f_{\KstP}\BF(\BdToJPsiKstP)\BF(\decay{\jpsi}{\mumu}) \\
\times&\BF(\decay{\KstP}{\Kp\pim})\frac{N'_{\kpimm}}{(1-F_{\rm S}^{\jpsi\Kstarz})N'_{\jpsi\Kstarz}}.
\end{split}
\label{eqn:dbfdq2}
\end{equation}
 
\noindent where $N'_{\kpimm}$ and $N'_{\jpsi\Kstarz}$ are the acceptance corrected yields of the \BdToKpimm and \BdToJPsiKst candidates, respectively. This yield of \BdToJPsiKst candidates has to be corrected for the S-wave fraction within the narrow $\mkpi$ window of \BdToJPsiKst decays, $F_{\rm S}^{\jpsi\Kstarz}$. The value of $F_{\rm S}^{\jpsi\Kstarz}$ is obtained from Ref.~\cite{LHCb-JpsiKstar} and is adjusted to the $\mkpi$ range $796<\mkpi<996~\mevcc$. The branching fractions $\BF(\BdToJPsiKstP)$, $\BF(\decay{\jpsi}{\mumu})$ and $\BF(\decay{\KstP}{\Kp\pim})$ are $(1.19\pm0.01\pm0.08)\times10^{-3}$~\cite{belle-z-paper}, $(5.961 \pm 0.033) \times 10^{-2}$~\cite{pdg} and 2/3, respectively. The fraction $f_{\KstP}$ is used to scale the value of $\BF(\BdToJPsiKstP)$ to the correct \mkpi range and is calculated by integrating the $\KstP$ lineshape given in Ref.~\cite{belle-z-paper} over the range $796<\mkpi<996~\mevcc$.
 
\subsubsection{Acceptance corrected yields}
 
To avoid making any assumptions about the unknown distributions of the \BdToKpimm candidates, the event-by-event efficiencies described in Sec.~\ref{sec:kpimm:acceptance} are used to correct the measured yields by calculating the average acceptance weight, where each weight is the reciprocal of the event-by-event efficiency.
 
For the case where there are only signal candidates present, the average weight would simply be calculated as,
 
\begin{equation}
\overline{w} = \frac{1}{N}\sum\limits_{i}^{N}w_{i},
\label{eqn:average_eff}
\end{equation}
 
\noindent where $w_{i}$ is the event-by-event acceptance and $N$ is the number of candidates.  An estimate for the error on the average weight is given by,
 
\begin{equation}
\delta_{\overline{w}} = \sqrt{\frac{1}{N(N-1)}\sum\limits_{i}^{N}(w_{i}-\overline{w})^{2}}.
\end{equation}
 
Due to the presence of background, the average weight calculated in the signal mass window will be an admixture of the average weight for both signal candidates ($\overline{w}_{sig}$) and background candidates ($\overline{w}_{bkg}$),
 
\begin{equation}
\overline{w}_{mix} = \frac{N_{sig}\overline{w}_{sig} + N_{bkg}\overline{w}_{bkg}}{N_{sig}+N_{bkg}},
\end{equation}
 
\noindent where $N_{sig}$ and $N_{bkg}$ are the number of signal and background events in the signal mass window, respectively. This can be rearranged to give the average weight for the signal candidates,
 
\begin{equation}
\overline{w}_{sig} = \frac{(N_{sig}+N_{bkg})\overline{w}_{mix} -  N_{bkg}\overline{w}_{bkg}}{N_{sig}}.
\end{equation}
 
However, what is needed for both \BdToKpimm and \BdToJPsiKst is the acceptance corrected yield $\overline{w}_{sig}N_{sig}$.  This is given by,
 
\begin{equation}
\overline{w}_{sig}N_{sig} = (N_{sig}+N_{bkg})\overline{w}_{mix} -  N_{bkg}\overline{w}_{bkg}
\end{equation}
 
\noindent where the errors are propagated as,
 
\begin{equation}
\begin{split}
\sigma_{\overline{w}_{sig}N_{sig}}^{2} =~ &(N_{sig}+N_{bkg})^{2}\sigma_{\overline{w}_{mix}}^{2} + (-N_{bkg})^{2}\sigma_{\overline{w}_{bkg}}^{2}\\
&+(\overline{w}_{mix})^{2}\sigma_{N_{sig}}^{2} + (\overline{w}_{mix}-\overline{w}_{bkg})^{2}\sigma_{N_{bkg}}^{2}.
\end{split}
\end{equation}
 
The signal region is defined as $5230<\mkpimm<5330$~\mevcc and the background region as $5350<\mkpimm<5700$~\mevcc. For the resonant mode, the background region is altered to $5450<\mkpimm<5700$~\mevcc in order to prevent any potential pollution from \BdToJPsiKst or \BsToJPsiKst candidates.

\subsubsection{Toy studies}

Toy studies are performed for the extraction of $\overline{w}_{sig}N_{sig}$ with different numbers of signal and background candidates.  In each toy $N_{sig}$, $N_{bkg}$ are Poisson fluctuated.  The nominal mass models, described in Sec.~\ref{sec:kpimm:massfit}, are used to generate signal and background candidates. The weights for both signal and background are sampled from two gaussian functions with different means.  The pulls for the extraction of $\overline{w}_{sig}N_{sig}$ are shown in Fig.~\ref{fig:bf:pulls}. No bias is observed.
 
\begin{figure}[!tb]
 \centering
 \includegraphics[width=0.45\textwidth]{figs/kpimm/bf/n_prime_low_yield.pdf}
 \includegraphics[width=0.45\textwidth]{figs/kpimm/bf/n_prime_med_yield.pdf}
 \includegraphics[width=0.45\textwidth]{figs/kpimm/bf/n_prime_high_yield.pdf}
 \caption{Pull plots for the extraction of $\overline{w}_{sig}N_{sig}$ with different numbers of signal and background candidates.}
 \label{fig:bf:pulls}
\end{figure}

\subsubsection{Results}

The results for the differential branching fraction are given in Fig.~\ref{fig:bf}.  The uncertainties shown are the quadratic sum of the statistical and systematic uncertainties.  The results are also presented in Table~\ref{tab:bf}.  The various sources of the systematic uncertainties are described in Sec.~\ref{sec:kpimm:systematics}.
 
\begin{figure}[!tb]
\centering
\includegraphics[width=0.7\textwidth]{figs/kpimm/bf/dbfdq2.pdf}
\caption{Differential branching fraction of \BdToKpimm in bins of \qsq. The uncertainties shown are the quadratic sum of the statistical and systematic uncertainties.}
\label{fig:bf}
\end{figure}
 
\begin{table}[!tb]
\caption{Differential branching fraction of \BdToKpimm in bins of \qsq. The first uncertainty is statistical, the second systematic and the third due to the uncertainty on the \BdToJPsiKstP and $\decay{\jpsi}{\mumu}$ branching fractions.}
\label{tab:bf}
\begin{center}
\begin{tabular}{lc}
\qsq [$\gevgevcccc$] & $\deriv\BF/\deriv\qsq \times 10^{-8}~[c^{4}/\gev^{2}]$ \\
\hline
$[0.10,0.98]$ & 1.60 $\pm$ 0.28 $\pm$ 0.04 $\pm$ 0.11 \\
$[1.10,2.50]$ & 1.14 $\pm$ 0.19 $\pm$ 0.03 $\pm$ 0.08 \\
$[2.50,4.00]$ & 0.91 $\pm$ 0.16 $\pm$ 0.03 $\pm$ 0.06 \\
$[4.00,6.00]$ & 0.56 $\pm$ 0.12 $\pm$ 0.02 $\pm$ 0.04 \\
$[6.00,8.00]$ & 0.49 $\pm$ 0.11 $\pm$ 0.01 $\pm$ 0.03 \\
\hline
$[1.10,6.00]$ & 0.82 $\pm$ 0.09 $\pm$ 0.02 $\pm$ 0.06 \\
\end{tabular}
\end{center}
\end{table}