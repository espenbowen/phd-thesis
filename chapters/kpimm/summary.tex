\subsection{Summary}

This chapter presents the differential branching fraction and angular moments analysis of the decay $\decay{\B^0}{K^{+}\pi^{-}\mu^{+}\mu^{-}}$ in the $K^{+}\pi^{-}$ invariant mass range $1330<m(\Kp\pim)<1530\mathrm{\,Me\kern -0.1em V\!/}c^{2}$.  The dataset corresponds to an integrated luminosity of 3$\ensuremath{\mbox{\,fb}^{-1}}\xspace$ of $pp$ collision data collected at the LHCb experiment.  The differential branching fraction is reported in five narrow \qsq bins between 0.1 and 8.0\gevgevcccc and in the range $1.1<\qsq<6.0\gevgevcccc$, where an angular moments analysis is also performed.

The measured values of the angular observables $\overline{\Gamma}_{2}$ and $\overline{\Gamma}_{3}$ point towards the presence of large interference effects between the S- and P- or D-wave contributions. Using only $\overline{\Gamma}_{5}$ and $\overline{\Gamma}_{10}$ it is possible to estimate the D-wave fraction, $F_{\rm D}$,  yielding an upper limit of $F_{\rm D}<0.29$ at 95\% confidence level. This value is lower than naively expected from amplitude analyses of \BdToJPsiKpi decays~\cite{belle-z-paper}.

With a given prediction for the form-factors, the underlying Wilson coefficients can be extracted from a least squares fit to the normalised moments and covariance matrix. While first estimates for the form factors are given in Ref.~\cite{lu-wang}, no interpretation of the results in terms of the Wilson coefficients is made at this time.  It is hoped that this measurement will stimulate theory effort since it can provide an important contribution to understanding the pattern of deviations with respect to SM predictions, which has been observed in other $b\to s\mu\mu$ transitions.