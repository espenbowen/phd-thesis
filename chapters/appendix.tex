{\noindent\normalfont\bfseries\Large Appendices}

\appendix

% \section{Hough transforms}
% \label{sec:appendix:hough}
% \clearpage

\section{Angular distribution}
\label{sec:appendix:angular-distribution}

The transversity-basis moments of the 41 orthonormal angular functions defined in Eq.~\ref{eqn:vector_moments}
 are shown in Table~\ref{table:spd_mom_trans}. The orthonormal angular basis is constructed out of spherical harmonics, \mbox{$Y^m_l \equiv Y^m_l (\thetal,\phi)$}, and reduced spherical harmonics, \mbox{${P^m_l \equiv \sqrt{2 \pi}Y^m_l(\thetak,0)}$}. The S-, P- and D-wave transversity amplitudes are denoted as $S^{\{L,R\}}$, $H^{\{L,R\}}_{\{0,\parallel,\perp\}}$ and $D^{\{L,R\}}_{\{0,\parallel,\perp\}}$, respectively. The indices $L$ and $R$ denote the (left- and right-handed) chirality of the lepton system.
 
\begin{table}[!tb]
\centering
\caption{The transversity-basis moments of the 41 orthonormal angular functions $f_i(\Omega)$ in Eq.~\ref{eqn:vector_moments}.}
\label{table:spd_mom_trans}
\resizebox{\textwidth}{!}{
\begin{tabular}{c|c|c|c} 
 $i$    &   $f_i(\Omega)$             & $\Gamma^{L, {\rm tr}}_i(\qsq)$ & $\eta^{L\to R}_i$  \\ \hline \hline
 1   &   $P^0_0 Y^0_0$     &  $\left[ \hzsq + \hpasq + \hpesq + \ssq + \dzsq + \dpasq + \dpesq\right]$ & + ($L \to R$)\\ \hline 
 2   &   $P^0_1 Y^0_0$     &  $2\left[\frac{2}{\sqrt{5}} \rhzdz + \rshz + \sqrt{\frac{3}{5}}  \rel( H^L_\parallel D^{L\ast}_\parallel + H^L_\perp D^{L\ast}_\perp  )\right]$ & " \\ \hline 
 3   &   $P^0_2 Y^0_0$     &  $\frac{\sqrt{5}}{7}$ (\dpasq + \dpesq) - $\frac{1}{\sqrt{5}}$ (\hpasq + \hpesq) + $\frac{2}{\sqrt{5}}$ \hzsq  + $\frac{10}{7\sqrt{5}}$ \dzsq + $2$ \rsdz & " \\  \hline
 4   &   $P^0_3 Y^0_0$     &  $\frac{6}{\sqrt{35}} \left[ - \rel(H^L_\parallel D^{L\ast}_\parallel +  H^L_\perp D^{L\ast}_\perp)  + \sqrt{3} \rhzdz  \right]$ & "\\  \hline
 5   &   $P^0_4 Y^0_0$     &  $\frac{2}{7} \left[ -2 (\dpasq + \dpesq) + 3 \dzsq \right] $ & "\\  \hline
 6   &   $P^0_0 Y^0_2$     &  $\frac{1}{2 \sqrt{5}} \left[ (\dpasq + \dpesq) + (\hpasq + \hpesq) - 2 \ssq - 2 \dzsq - 2 \hzsq \right]$ & " \\  \hline
 7   &   $P^0_1 Y^0_2$     &  $\left[ \frac{\sqrt{3}}{5} \rel(H^L_\parallel D^{L\ast}_\parallel  + H^L_\perp D^{L\ast}_\perp) - \frac{2}{\sqrt{5}} \rel(S^L H^{L\ast}_0)  - \frac{4}{5} \rel(H^L_0 D^{L\ast}_0)\right] $ & "  \\ \hline
 8   &   $P^0_2 Y^0_2$     &  $ \left[ \frac{1}{14} (\dpasq + \dpesq) - \frac{2}{7} \dzsq - \frac{1}{10} (\hpasq + \hpesq) - \frac{2}{5} \hzsq - \frac{2}{\sqrt{5}} \rsdz \right]$ & "  \\  \hline
 9   &   $P^0_3 Y^0_2$     &  $ - \frac{3}{5 \sqrt{7}} \left[ \rel( H^L_\parallel D^{L \ast}_\parallel + H^L_\perp D^{L \ast}_\perp) + 2 \sqrt{3} \rel(H^L_0 D^{L \ast}_0 ) \right] $ & "\\  \hline
 10  &   $P^0_4 Y^0_2$     &  $ -\frac{2}{7 \sqrt{5}}  \left[ \dpasq + \dpesq + 3 \dzsq \right] $ & "  \\  \hline
 11  &   $P^1_1 \sqrt{2}\rel(Y^1_2)$ &  $-\frac{3}{\sqrt{10}} \left[ \sqrt{\frac{2}{3}} \rel(H^L_\parallel S^{L \ast}) - \sqrt{\frac{2}{15}} \rel(H^L_\parallel D^{L \ast}_0  ) + \sqrt{\frac{2}{5}} \rel(D^L_\parallel H^{L \ast}_0 ) \right] $  & "\\  \hline
 12  &   $P^1_2 \sqrt{2}\rel(Y^1_2)$ &  $-\frac{3}{5} \left[ \rel( H^L_\parallel H^{L \ast}_0)  + \sqrt{\frac{5}{3}} \rel (D^L_\parallel S^{L \ast})  + \frac{5}{7 \sqrt{3}} \rel(D^L_\parallel D^{L\ast}_0) \right] $ & " \\  \hline
 13  &   $P^1_3 \sqrt{2}\rel(Y^1_2)$ &  $-\frac{6}{5 \sqrt{14}} \left[2 \rel(D^L_\parallel H^{L\ast}_0)  + \sqrt{3} \rel(H^L_\parallel D^{L\ast}_0 ) \right] $ & " \\  \hline
 14  &   $P^1_4 \sqrt{2}\rel(Y^1_2)$ &  $- \frac{6}{7\sqrt{2}} \rel(D^L_\parallel D^{L\ast}_0)$  & " \\  \hline
 15  &   $P^1_1 \sqrt{2}\img(Y^1_2)$ &  $3 \left[ \frac{1}{\sqrt{15}} \img(H^L_\perp S^{L\ast}) + \frac{1}{5} \img(D^L_\perp H^{L\ast}_0) - \frac{1}{5 \sqrt{3}}  \img(H^L_\perp D^{L\ast}_0) \right]  $  & " \\  \hline
 16  &   $P^1_2 \sqrt{2}\img(Y^1_2)$ &  $ 3\left[ \frac{1}{7 \sqrt{3}} \img(D^L_\perp D^{L\ast}_0)  + \frac{1}{5} \img(H^L_\perp H^{L\ast}_0)  + \frac{1}{\sqrt{15}} \img(D^L_\perp S^{L\ast})   \right] $  & " \\  \hline
 17  &   $P^1_3 \sqrt{2}\img(Y^1_2)$ &  $\frac{6}{5 \sqrt{14}} \left[ 2 \img(D^L_\perp H^{L\ast}_0)  + \sqrt{3} \img(H^L_\perp D^{L\ast}_0) \right]  $   & " \\  \hline
 18  &   $P^1_4 \sqrt{2}\img(Y^1_2)$ &  $\frac{6}{7\sqrt{2}} \img(D^L_\perp D^{L\ast}_0)$  & " \\  \hline
 19  &   $P^0_0 \sqrt{2}\rel(Y^2_2)$ &  $-\frac{3}{2\sqrt{15}} \left[ (\hpasq - \hpesq) + (\dpasq - \dpesq) \right] $  & " \\  \hline
 20  &   $P^0_1 \sqrt{2}\rel(Y^2_2)$ &  $-\frac{3}{5} \left[ \rel(H^L_\parallel D^{L\ast}_\parallel)   - \rel(D^L_\perp H^{L\ast}_\perp) \right] $  & " \\  \hline
 21  &   $P^0_2 \sqrt{2}\rel(Y^2_2)$ &  $\frac{\sqrt{3}}{2} \left[ - \frac{1}{7} (\dpasq - \dpesq)   + \frac{1}{5} ( \hpasq - \hpesq ) \right] $  & " \\  \hline
 22  &   $P^0_3 \sqrt{2}\rel(Y^2_2)$ &  $\frac{3}{5} \sqrt{ \frac{3}{7}} \left[ \rel(H^L_\parallel D^{L\ast}_\parallel)   - \rel(D^L_\perp H^{L\ast}_\perp) \right] $  & " \\  \hline
 23  &   $P^0_4 \sqrt{2}\rel(Y^2_2)$ &  $\frac{2}{7} \sqrt{ \frac{3}{5}}  (\dpasq - \dpesq) $ & " \\  \hline
 24  &   $P^0_0 \sqrt{2}\img(Y^2_2)$ &  $\sqrt{\frac{3}{5}} \left[ \img(H^L_\perp H^{L\ast}_\parallel) + \img(D^L_\perp D^{L\ast}_\parallel) \right] $   & " \\  \hline
 25  &   $P^0_1 \sqrt{2}\img(Y^2_2)$ &  $\frac{3}{5} \img(  H^L_\perp D^{L\ast}_\parallel +  D^L_\perp H^{L\ast}_\parallel )  $  & " \\ \hline
 26  &   $P^0_2 \sqrt{2}\img(Y^2_2)$ &  $ \sqrt{3} \left[\frac{1}{7} \img(D^L_\perp D^{L\ast}_\parallel)   - \frac{1}{5} \img(H^L_\perp H^{L\ast}_\parallel)\right] $  & " \\ \hline
 27  &   $P^0_3 \sqrt{2}\img(Y^2_2)$ &  $-\frac{3}{5} \sqrt{ \frac{3}{7}}  \img(D^L_\perp H^{L\ast}_\parallel + H^L_\perp D^{L\ast}_\parallel)  $  & " \\ \hline
 28  &   $P^0_4 \sqrt{2}\img(Y^2_2)$ &  $-\frac{4}{7} \sqrt{\frac{3}{5}}  \img(D^L_\perp D^{L\ast}_\parallel) $   & " \\ \hline \hline
 29  &   $P^0_0 Y^0_1$     &  $-\sqrt{3}\left[ \rel(H^L_\perp H^{L\ast}_\parallel) + \rel(D^L_\perp D^{L\ast}_\parallel) \right]$  & - ($L \to R$) \\ \hline
 30  &   $P^0_1 Y^0_1$     &  $-\frac{3}{\sqrt{5}} \rel( H^L_\perp D^{L\ast}_\parallel + H^L_\parallel D^{L\ast}_\perp ) $ & " \\ \hline
 31  &   $P^0_2 Y^0_1$     &  $-\frac{3}{\sqrt{15}} \left[ \frac{5}{7} \rel(D^L_\perp D^{L\ast}_\parallel) - \rel(H^L_\perp H^{L\ast}_\parallel)  \right]$ & " \\ \hline
 32  &   $P^0_3 Y^0_1$     &  $\frac{9}{\sqrt{105}}  \rel(H^L_\perp D^{L\ast}_\parallel  + H^L_\parallel D^{L\ast}_\perp ) $ & " \\ \hline
 33  &   $P^0_4 Y^0_1$     &  $\frac{4\sqrt{3}}{7} \rel(D^L_\perp D^{L\ast}_\parallel)$  & " \\ \hline
 34  &   $P^1_1 \sqrt{2}\rel(Y^1_1)$   & $\sqrt{\frac{3}{5}} \left[ \sqrt{5} \rel(H^L_\perp S^{L \ast})  + \sqrt{3} \rel(D^L_\perp H^{L\ast}_0)  - \rel(H^L_\perp D^{L \ast}_0) \right]$  & " \\ \hline
 35  &   $P^1_2 \sqrt{2}\rel(Y^1_1)$   & $ 3 \left[ \frac{1}{\sqrt{5}} \rel(H^L_\perp H^{L \ast}_0)  + \frac{1}{\sqrt{3}} \rel(D^L_\perp S^{L\ast})  + \frac{5}{21} \sqrt{\frac{3}{5}} \rel(D^L_\perp D^{L \ast}_0 ) \right] $  & " \\ \hline
 36  &   $P^1_3 \sqrt{2}\rel(Y^1_1)$   & $ \frac{6}{\sqrt{70}} \left[ 2 \rel(D^L_\perp H^{L \ast}_0)  + \sqrt{3} \rel(H^L_\perp D^{L\ast}_0) \right]$  & " \\ \hline
 37  &   $P^1_4 \sqrt{2}\rel(Y^1_1)$   & $\frac{3 \sqrt{10}}{7} \rel(D^L_\perp D^{L \ast}_0 ) $  & " \\ \hline
 38  &   $P^1_1 \sqrt{2}\img(Y^1_1)$   & $-\sqrt{\frac{3}{5}} \left[ \sqrt{5} \img ( H^L_\parallel S^{L\ast}) + \sqrt{3} \img(D^L_\parallel H^{L \ast}_0) - \img(H^L_\parallel D^{L \ast}_0)  \right]  $  & " \\ \hline
 39  &   $P^1_2 \sqrt{2}\img(Y^1_1)$   & $ -\sqrt{\frac{3}{5}} \left[ \sqrt{3} \img(H^L_\parallel H^{L \ast}_0)  + \sqrt{5} \img(D^L_\parallel S^{L\ast})  + \frac{5}{7} \img(D^L_\parallel D^{L \ast}_0 )\right] $  & " \\ \hline
 40  &   $P^1_3 \sqrt{2}\img(Y^1_1)$   & $ -6\sqrt{\frac{1}{70}}\left[ 2\img(D^L_\parallel H^{L \ast}_0)  + \sqrt{3} \img(H^L_\parallel D^{L\ast}_0)\right]$   & " \\ \hline
 41  &   $P^1_4 \sqrt{2}\img(Y^1_1)$   & $-\frac{3}{7} \sqrt{10} \img(D^L_\parallel D^{L\ast}_0) $  & " \\ \hline
\end{tabular}

}
\end{table}

\clearpage
\section{Agreement between data and simulation}
\label{sec:appendix:data-mc}

\subsection{PID resampling}
\label{sec:appendix:data-mc:pid}

\begin{figure}[!hb]
 \centering
 \includegraphics[width=0.49\textwidth]{figs/kpimm/data-mc/resampling/Muplus_PIDmu.pdf}
 \includegraphics[width=0.49\textwidth]{figs/kpimm/data-mc/resampling/Muminus_PIDmu.pdf}
 \includegraphics[width=0.49\textwidth]{figs/kpimm/data-mc/resampling/K_PIDmu.pdf}
 \includegraphics[width=0.49\textwidth]{figs/kpimm/data-mc/resampling/Pi_PIDmu.pdf}
 \includegraphics[width=0.49\textwidth]{figs/kpimm/data-mc/resampling/Pi_PIDp.pdf}
 \caption{Data-simulation agreement for the PID variables used in the selection of \BdToKpimm. The black data points show the distributions for background-subtracted \BdToJPsiKst decays in data. The red dashed histograms show the nominal distribution for simulated \BdToJPsiKst decays. The blue histograms show the distribution for simulated \BdToJPsiKst decays after the resampling procedure.}
\label{fig:appendix:data-mc:pid}
\end{figure}

\clearpage

\subsection{Data-MC agreement for BDT input variables}
\label{sec:appendix:data-mc-bdtvars}
 
\begin{figure}[!hb]
 \centering
 \includegraphics[width=0.32\textwidth]{figs/kpimm/data-mc/bdt/B0_DiraAngle.pdf}
 \includegraphics[width=0.32\textwidth]{figs/kpimm/data-mc/bdt/B0_TAU.pdf}
 \includegraphics[width=0.32\textwidth]{figs/kpimm/data-mc/bdt/B0_ENDVERTEX_CHI2.pdf}
 \includegraphics[width=0.32\textwidth]{figs/kpimm/data-mc/bdt/B0_P.pdf}
 \includegraphics[width=0.32\textwidth]{figs/kpimm/data-mc/bdt/B0_PT.pdf}
 \includegraphics[width=0.32\textwidth]{figs/kpimm/data-mc/bdt/Muplus_isolation_V2_15.pdf}
 \includegraphics[width=0.32\textwidth]{figs/kpimm/data-mc/bdt/Muminus_isolation_V2_15.pdf}
 \includegraphics[width=0.32\textwidth]{figs/kpimm/data-mc/bdt/kaon_isolation.pdf}
 \includegraphics[width=0.32\textwidth]{figs/kpimm/data-mc/bdt/pion_isolation.pdf}
 \includegraphics[width=0.32\textwidth]{figs/kpimm/data-mc/bdt/K_PIDK.pdf}
 \includegraphics[width=0.32\textwidth]{figs/kpimm/data-mc/bdt/Pi_PIDK.pdf}
 \includegraphics[width=0.32\textwidth]{figs/kpimm/data-mc/bdt/Muplus_PIDmu.pdf}
 \includegraphics[width=0.32\textwidth]{figs/kpimm/data-mc/bdt/Muminus_PIDmu.pdf}
 
 \caption{Data-simulation agreement for the variables for each of the variables used as input to the BDT. The black data points show the distributions for background-subtracted \BdToJPsiKst decays in data. The blue dashed histograms show the distribution for resampled, simulated \BdToJPsiKst decays. The green histograms show the distribution for resampled, simulated \BdToJPsiKst decays with the candidate weights applied.}
 \label{fig:appendix:data-mc:bdtvars}
\end{figure}

\clearpage
\section{Acceptance correction}
\label{sec:appendix:acceptance}

In order to study the correlations between the kinematic variables used in the acceptance correction it is useful to look at two dimensional distributions.  Figures~\ref{fig:acceptance:2d_1}$-$\ref{fig:acceptance:2d_4} show the two dimensional distributions for each of the pairs of variables.  The plots on the left show the distributions for the simulated decays used to determine the acceptance correction.  The plots on the right show the distributions for toy events generated flat in each of the variables and weighted by the efficiency determined from the acceptance parameterisation. Good agreement is found between the two distributions.

\begin{figure}[!b]
 \centering
 \includegraphics[width=0.9\textwidth]{figs/kpimm/acceptance/2d/q2_ctl.pdf}
 \includegraphics[width=0.9\textwidth]{figs/kpimm/acceptance/2d/q2_ctk.pdf}
 \caption{Two dimensional distributions of the acceptance parameterisation.}
 \label{fig:acceptance:2d_1}
\end{figure}
 
\begin{figure}[!tb]
 \centering
 \includegraphics[width=0.9\textwidth]{figs/kpimm/acceptance/2d/q2_phi.pdf}
 \includegraphics[width=0.9\textwidth]{figs/kpimm/acceptance/2d/q2_mkpi.pdf}
 \includegraphics[width=0.9\textwidth]{figs/kpimm/acceptance/2d/ctl_ctk.pdf}
 \caption{Two dimensional distributions of the acceptance parameterisation.}
 \label{fig:acceptance:2d_2}
\end{figure}

\begin{figure}[!tb]
 \centering
 \includegraphics[width=0.9\textwidth]{figs/kpimm/acceptance/2d/ctl_phi.pdf}
 \includegraphics[width=0.9\textwidth]{figs/kpimm/acceptance/2d/ctl_mkpi.pdf}
 \includegraphics[width=0.9\textwidth]{figs/kpimm/acceptance/2d/ctk_phi.pdf}
 \caption{Two dimensional distributions of the acceptance parameterisation.}
 \label{fig:acceptance:2d_3}
\end{figure}

\begin{figure}[!tb]
 \centering
 \includegraphics[width=0.9\textwidth]{figs/kpimm/acceptance/2d/ctk_mkpi.pdf}
 \includegraphics[width=0.9\textwidth]{figs/kpimm/acceptance/2d/phi_mkpi.pdf}
 \caption{Two dimensional distributions of the acceptance parameterisation.}
 \label{fig:acceptance:2d_4}
\end{figure}

\clearpage
\section{The \boldmath{\mkpimm} invariant mass distribution}
\label{sec:appendix:massfit}

Figure~\ref{fig:appendix:massfit:bins} shows the fits to the \mkpimm distribution in each of the \qsq bins used for the differential branching fraction measurement.
 
\begin{figure}[!hb]
\centering
\includegraphics[width=0.48\linewidth]{figs/kpimm/massfit/fitKpimumu_q2_0p1_0p98.pdf}
\includegraphics[width=0.48\linewidth]{figs/kpimm/massfit/fitKpimumu_q2_1p1_2p5.pdf}
\includegraphics[width=0.48\linewidth]{figs/kpimm/massfit/fitKpimumu_q2_2p5_4p0.pdf}
\includegraphics[width=0.48\linewidth]{figs/kpimm/massfit/fitKpimumu_q2_4p0_6p0.pdf}
\includegraphics[width=0.48\linewidth]{figs/kpimm/massfit/fitKpimumu_q2_6p0_8p0.pdf}
\includegraphics[width=0.48\linewidth]{figs/kpimm/massfit/fitKpimumu_q2_1p1_6p0.pdf}
 
\caption{Invariant mass \mkpimm for the signal decay \BdToKpimm in  each of the \qsq bins used for the differential branching fraction measurement. The solid black line represents the total fitted function.  The individual components of the signal (blue shaded area) and combinatorial background (red hatched area) are also shown.}
\label{fig:appendix:massfit:bins}
\end{figure}

\clearpage
\section{Toy studies for the angular moments analysis}
\label{sec:appendix:kpimm:toys}

The results of the pull studies described in Sec.~\ref{sec:kpimm:angular-analysis:toys} are shown in Fig.~\ref{fig:appendix:kpimm:angular-analysis:toys:pulls:1} and Fig.~\ref{fig:appendix:kpimm:angular-analysis:toys:pulls:2}. No bias is observed for any of the measured moments and the statistical errors are correctly evaluated.

\begin{sidewaysfigure}
\centering
\includegraphics[width=0.24\textwidth]{figs/kpimm/angular-analysis/toys/pull_m_2.pdf}
\includegraphics[width=0.24\textwidth]{figs/kpimm/angular-analysis/toys/pull_m_3.pdf}
\includegraphics[width=0.24\textwidth]{figs/kpimm/angular-analysis/toys/pull_m_4.pdf}
\includegraphics[width=0.24\textwidth]{figs/kpimm/angular-analysis/toys/pull_m_5.pdf}
\includegraphics[width=0.24\textwidth]{figs/kpimm/angular-analysis/toys/pull_m_6.pdf}
\includegraphics[width=0.24\textwidth]{figs/kpimm/angular-analysis/toys/pull_m_7.pdf}
\includegraphics[width=0.24\textwidth]{figs/kpimm/angular-analysis/toys/pull_m_8.pdf}
\includegraphics[width=0.24\textwidth]{figs/kpimm/angular-analysis/toys/pull_m_9.pdf}
\includegraphics[width=0.24\textwidth]{figs/kpimm/angular-analysis/toys/pull_m_10.pdf}
\includegraphics[width=0.24\textwidth]{figs/kpimm/angular-analysis/toys/pull_m_11.pdf}
\includegraphics[width=0.24\textwidth]{figs/kpimm/angular-analysis/toys/pull_m_12.pdf}
\includegraphics[width=0.24\textwidth]{figs/kpimm/angular-analysis/toys/pull_m_13.pdf}
\includegraphics[width=0.24\textwidth]{figs/kpimm/angular-analysis/toys/pull_m_14.pdf}
\includegraphics[width=0.24\textwidth]{figs/kpimm/angular-analysis/toys/pull_m_15.pdf}
\includegraphics[width=0.24\textwidth]{figs/kpimm/angular-analysis/toys/pull_m_16.pdf}
\includegraphics[width=0.24\textwidth]{figs/kpimm/angular-analysis/toys/pull_m_17.pdf}
\includegraphics[width=0.24\textwidth]{figs/kpimm/angular-analysis/toys/pull_m_18.pdf}
\includegraphics[width=0.24\textwidth]{figs/kpimm/angular-analysis/toys/pull_m_19.pdf}
\includegraphics[width=0.24\textwidth]{figs/kpimm/angular-analysis/toys/pull_m_20.pdf}
\includegraphics[width=0.24\textwidth]{figs/kpimm/angular-analysis/toys/pull_m_21.pdf}
\caption{Pull distributions of the normalised moments, $\overline{\Gamma}_{i}$, for simulated datasets.}
\label{fig:appendix:kpimm:angular-analysis:toys:pulls:1}
\end{sidewaysfigure}

\begin{sidewaysfigure}
\centering
\includegraphics[width=0.24\textwidth]{figs/kpimm/angular-analysis/toys/pull_m_22.pdf}
\includegraphics[width=0.24\textwidth]{figs/kpimm/angular-analysis/toys/pull_m_23.pdf}
\includegraphics[width=0.24\textwidth]{figs/kpimm/angular-analysis/toys/pull_m_24.pdf}
\includegraphics[width=0.24\textwidth]{figs/kpimm/angular-analysis/toys/pull_m_25.pdf}
\includegraphics[width=0.24\textwidth]{figs/kpimm/angular-analysis/toys/pull_m_26.pdf}
\includegraphics[width=0.24\textwidth]{figs/kpimm/angular-analysis/toys/pull_m_27.pdf}
\includegraphics[width=0.24\textwidth]{figs/kpimm/angular-analysis/toys/pull_m_28.pdf}
\includegraphics[width=0.24\textwidth]{figs/kpimm/angular-analysis/toys/pull_m_29.pdf}
\includegraphics[width=0.24\textwidth]{figs/kpimm/angular-analysis/toys/pull_m_30.pdf}
\includegraphics[width=0.24\textwidth]{figs/kpimm/angular-analysis/toys/pull_m_31.pdf}
\includegraphics[width=0.24\textwidth]{figs/kpimm/angular-analysis/toys/pull_m_32.pdf}
\includegraphics[width=0.24\textwidth]{figs/kpimm/angular-analysis/toys/pull_m_33.pdf}
\includegraphics[width=0.24\textwidth]{figs/kpimm/angular-analysis/toys/pull_m_34.pdf}
\includegraphics[width=0.24\textwidth]{figs/kpimm/angular-analysis/toys/pull_m_35.pdf}
\includegraphics[width=0.24\textwidth]{figs/kpimm/angular-analysis/toys/pull_m_36.pdf}
\includegraphics[width=0.24\textwidth]{figs/kpimm/angular-analysis/toys/pull_m_37.pdf}
\includegraphics[width=0.24\textwidth]{figs/kpimm/angular-analysis/toys/pull_m_38.pdf}
\includegraphics[width=0.24\textwidth]{figs/kpimm/angular-analysis/toys/pull_m_39.pdf}
\includegraphics[width=0.24\textwidth]{figs/kpimm/angular-analysis/toys/pull_m_40.pdf}
\includegraphics[width=0.24\textwidth]{figs/kpimm/angular-analysis/toys/pull_m_41.pdf}
\caption{Pull distributions of the normalised moments, $\overline{\Gamma}_{i}$, for simulated datasets.}
\label{fig:appendix:kpimm:angular-analysis:toys:pulls:2}
\end{sidewaysfigure}

\clearpage
\section{Peaking background systematic}
\label{sec:appendix:kernel}

The effect of residual peaking background contributions is evaluated using pseudoexperiments where peaking background components are generated in addition to the signal and the combinatorial background. The peaking background candidates are first selected from data by isolating the decays using specific selections. In the following section, the method is shown explicity for the case of \BdToJPsiKpi decays with a $\pi\leftrightarrow\mu$ swap. 

As described in Sec.~\ref{sec:selection:exclusive}, candidates from \BdToJPsiKpi decays can contribute a background if the \pim (\Kp) is misidentified as a \mun (\mup) and the \mun (\mup) is misidentified as a \pim (\Kp).  For the case of $\mun \leftrightarrow \pim$ swaps, the invariant mass of the \pim and the \mup, after assigning the \pim the \muon mass hypothesis, should be consistent with the known \jpsi mass. This new mass hypothesis is denoted $m_{(\pi\to\mu)\mu}$.

Candidates are selected from data by requiring the $m_{(\pi\to\mu)\mu}$ invariant mass to be within $\pm200$\mevcc of the known \jpsi mass. The $m_{(\pi\to\mu)\mu}$ distribution is fitted using a Gaussian to model the \BdToJPsiKpi contribution and a second order Chebyshev polynomial to model the non-resonant contribution, as shown in Fig.~\ref{fig:appendix::kernel:massfit}. The \sPlot technique~\cite{splot} is used to isolate the \BdToJPsiKpi contribution.

\begin{figure}[!tb]
\centering
\includegraphics[width=0.6\textwidth]{figs/kpimm/selection/jpsi_pimu_fit.pdf}
\caption{The $m_{(\pi\to\mu)\mu}$ distribution for \BdToJPsiKpi candidates with a $\pi\leftrightarrow\mu$ swap.}
\label{fig:appendix::kernel:massfit}
\end{figure}

The distributions of the decay angles for \BdToJPsiKpi decays with a $\pi\leftrightarrow\mu$ swap are shown in Fig.~\ref{fig:appendix::kernel:angles}. A kernel estimator~\cite{kernel} is used to model the distributions, as shown by the blue histogram in Fig.~\ref{fig:appendix::kernel:angles}. The resulting probability density function is used to generate the peaking background contributions for the pseudoexperiments.

\begin{figure}[!tb]
\centering
\includegraphics[width=0.45\textwidth]{figs/kpimm/selection/jpsi_pimu_costhetal.pdf}
\includegraphics[width=0.45\textwidth]{figs/kpimm/selection/jpsi_pimu_costhetak.pdf}
\includegraphics[width=0.45\textwidth]{figs/kpimm/selection/jpsi_pimu_phi.pdf}
\caption{Distributions of each of the decay angles for \BdToJPsiKpi decays with a $\pi\leftrightarrow\mu$ swap. The distributions derived from the kernel estimator are shown by the blue histograms.}
\label{fig:appendix::kernel:angles}
\end{figure}